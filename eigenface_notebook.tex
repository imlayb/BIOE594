\documentclass[]{article}
\usepackage{lmodern}
\usepackage{amssymb,amsmath}
\usepackage{ifxetex,ifluatex}
\usepackage{fixltx2e} % provides \textsubscript
\ifnum 0\ifxetex 1\fi\ifluatex 1\fi=0 % if pdftex
  \usepackage[T1]{fontenc}
  \usepackage[utf8]{inputenc}
\else % if luatex or xelatex
  \ifxetex
    \usepackage{mathspec}
  \else
    \usepackage{fontspec}
  \fi
  \defaultfontfeatures{Ligatures=TeX,Scale=MatchLowercase}
\fi
% use upquote if available, for straight quotes in verbatim environments
\IfFileExists{upquote.sty}{\usepackage{upquote}}{}
% use microtype if available
\IfFileExists{microtype.sty}{%
\usepackage{microtype}
\UseMicrotypeSet[protrusion]{basicmath} % disable protrusion for tt fonts
}{}
\usepackage[margin=1in]{geometry}
\usepackage{hyperref}
\hypersetup{unicode=true,
            pdftitle={Eigenface Notebook},
            pdfborder={0 0 0},
            breaklinks=true}
\urlstyle{same}  % don't use monospace font for urls
\usepackage{color}
\usepackage{fancyvrb}
\newcommand{\VerbBar}{|}
\newcommand{\VERB}{\Verb[commandchars=\\\{\}]}
\DefineVerbatimEnvironment{Highlighting}{Verbatim}{commandchars=\\\{\}}
% Add ',fontsize=\small' for more characters per line
\usepackage{framed}
\definecolor{shadecolor}{RGB}{248,248,248}
\newenvironment{Shaded}{\begin{snugshade}}{\end{snugshade}}
\newcommand{\KeywordTok}[1]{\textcolor[rgb]{0.13,0.29,0.53}{\textbf{#1}}}
\newcommand{\DataTypeTok}[1]{\textcolor[rgb]{0.13,0.29,0.53}{#1}}
\newcommand{\DecValTok}[1]{\textcolor[rgb]{0.00,0.00,0.81}{#1}}
\newcommand{\BaseNTok}[1]{\textcolor[rgb]{0.00,0.00,0.81}{#1}}
\newcommand{\FloatTok}[1]{\textcolor[rgb]{0.00,0.00,0.81}{#1}}
\newcommand{\ConstantTok}[1]{\textcolor[rgb]{0.00,0.00,0.00}{#1}}
\newcommand{\CharTok}[1]{\textcolor[rgb]{0.31,0.60,0.02}{#1}}
\newcommand{\SpecialCharTok}[1]{\textcolor[rgb]{0.00,0.00,0.00}{#1}}
\newcommand{\StringTok}[1]{\textcolor[rgb]{0.31,0.60,0.02}{#1}}
\newcommand{\VerbatimStringTok}[1]{\textcolor[rgb]{0.31,0.60,0.02}{#1}}
\newcommand{\SpecialStringTok}[1]{\textcolor[rgb]{0.31,0.60,0.02}{#1}}
\newcommand{\ImportTok}[1]{#1}
\newcommand{\CommentTok}[1]{\textcolor[rgb]{0.56,0.35,0.01}{\textit{#1}}}
\newcommand{\DocumentationTok}[1]{\textcolor[rgb]{0.56,0.35,0.01}{\textbf{\textit{#1}}}}
\newcommand{\AnnotationTok}[1]{\textcolor[rgb]{0.56,0.35,0.01}{\textbf{\textit{#1}}}}
\newcommand{\CommentVarTok}[1]{\textcolor[rgb]{0.56,0.35,0.01}{\textbf{\textit{#1}}}}
\newcommand{\OtherTok}[1]{\textcolor[rgb]{0.56,0.35,0.01}{#1}}
\newcommand{\FunctionTok}[1]{\textcolor[rgb]{0.00,0.00,0.00}{#1}}
\newcommand{\VariableTok}[1]{\textcolor[rgb]{0.00,0.00,0.00}{#1}}
\newcommand{\ControlFlowTok}[1]{\textcolor[rgb]{0.13,0.29,0.53}{\textbf{#1}}}
\newcommand{\OperatorTok}[1]{\textcolor[rgb]{0.81,0.36,0.00}{\textbf{#1}}}
\newcommand{\BuiltInTok}[1]{#1}
\newcommand{\ExtensionTok}[1]{#1}
\newcommand{\PreprocessorTok}[1]{\textcolor[rgb]{0.56,0.35,0.01}{\textit{#1}}}
\newcommand{\AttributeTok}[1]{\textcolor[rgb]{0.77,0.63,0.00}{#1}}
\newcommand{\RegionMarkerTok}[1]{#1}
\newcommand{\InformationTok}[1]{\textcolor[rgb]{0.56,0.35,0.01}{\textbf{\textit{#1}}}}
\newcommand{\WarningTok}[1]{\textcolor[rgb]{0.56,0.35,0.01}{\textbf{\textit{#1}}}}
\newcommand{\AlertTok}[1]{\textcolor[rgb]{0.94,0.16,0.16}{#1}}
\newcommand{\ErrorTok}[1]{\textcolor[rgb]{0.64,0.00,0.00}{\textbf{#1}}}
\newcommand{\NormalTok}[1]{#1}
\usepackage{graphicx,grffile}
\makeatletter
\def\maxwidth{\ifdim\Gin@nat@width>\linewidth\linewidth\else\Gin@nat@width\fi}
\def\maxheight{\ifdim\Gin@nat@height>\textheight\textheight\else\Gin@nat@height\fi}
\makeatother
% Scale images if necessary, so that they will not overflow the page
% margins by default, and it is still possible to overwrite the defaults
% using explicit options in \includegraphics[width, height, ...]{}
\setkeys{Gin}{width=\maxwidth,height=\maxheight,keepaspectratio}
\IfFileExists{parskip.sty}{%
\usepackage{parskip}
}{% else
\setlength{\parindent}{0pt}
\setlength{\parskip}{6pt plus 2pt minus 1pt}
}
\setlength{\emergencystretch}{3em}  % prevent overfull lines
\providecommand{\tightlist}{%
  \setlength{\itemsep}{0pt}\setlength{\parskip}{0pt}}
\setcounter{secnumdepth}{0}
% Redefines (sub)paragraphs to behave more like sections
\ifx\paragraph\undefined\else
\let\oldparagraph\paragraph
\renewcommand{\paragraph}[1]{\oldparagraph{#1}\mbox{}}
\fi
\ifx\subparagraph\undefined\else
\let\oldsubparagraph\subparagraph
\renewcommand{\subparagraph}[1]{\oldsubparagraph{#1}\mbox{}}
\fi

%%% Use protect on footnotes to avoid problems with footnotes in titles
\let\rmarkdownfootnote\footnote%
\def\footnote{\protect\rmarkdownfootnote}

%%% Change title format to be more compact
\usepackage{titling}

% Create subtitle command for use in maketitle
\newcommand{\subtitle}[1]{
  \posttitle{
    \begin{center}\large#1\end{center}
    }
}

\setlength{\droptitle}{-2em}

  \title{Eigenface Notebook}
    \pretitle{\vspace{\droptitle}\centering\huge}
  \posttitle{\par}
    \author{}
    \preauthor{}\postauthor{}
    \date{}
    \predate{}\postdate{}
  
\usepackage{booktabs}
\usepackage{longtable}
\usepackage{array}
\usepackage{multirow}
\usepackage{wrapfig}
\usepackage{float}
\usepackage{colortbl}
\usepackage{pdflscape}
\usepackage{tabu}
\usepackage{threeparttable}
\usepackage{threeparttablex}
\usepackage[normalem]{ulem}
\usepackage{makecell}
\usepackage{xcolor}

\begin{document}
\maketitle

\begin{Shaded}
\begin{Highlighting}[]
\KeywordTok{source}\NormalTok{(}\StringTok{"eigenfaces.R"}\NormalTok{)}
\KeywordTok{library}\NormalTok{(plyr)}
\KeywordTok{library}\NormalTok{(knitr)}
\KeywordTok{library}\NormalTok{(kableExtra)}
\KeywordTok{library}\NormalTok{(ggplot2)}
\KeywordTok{library}\NormalTok{(gridExtra)}
\KeywordTok{library}\NormalTok{(factoextra)}
\KeywordTok{library}\NormalTok{(caret)}
\end{Highlighting}
\end{Shaded}

\subsection{Introduction}\label{introduction}

The data consists of 3993 pictures of human faces at a resolution of 128
x 128 in 8-bit grayscale. Data can be obtained from
\href{http://courses.media.mit.edu/2004fall/mas622j/04.projects/faces/}{MIT}.

\subsection{Data Import}\label{data-import}

The eigenface rawdata is packaged with this repository and can be loaded
with the following code. Two pictures were remove from the rawdata as
provided from MIT because they were of incorrect dimensions: 2416 and
2412. Moreover, 22 faces are removed for having no parsed information.

\begin{Shaded}
\begin{Highlighting}[]
\KeywordTok{source}\NormalTok{(}\StringTok{"eigenfaces.R"}\NormalTok{)}
\NormalTok{dat<-}\KeywordTok{importFaceMatrix}\NormalTok{()}
\end{Highlighting}
\end{Shaded}

Each picture is represented as vector of length
128\textsuperscript{2}=16,384. This vector can be reshaped into a
raster, which can be colored with a color scale --- grayscale in our
case. Three random faces have been generated below.

\begin{Shaded}
\begin{Highlighting}[]
\CommentTok{#This code may be run multiple times to see the huge diversity of images in the data.}
\KeywordTok{par}\NormalTok{(}\DataTypeTok{mfrow=}\KeywordTok{c}\NormalTok{(}\DecValTok{1}\NormalTok{,}\DecValTok{3}\NormalTok{))}
\NormalTok{selected<-}\KeywordTok{sample}\NormalTok{(}\KeywordTok{rownames}\NormalTok{(dat),}\DecValTok{3}\NormalTok{)}
\NormalTok{\{}\KeywordTok{plotImage}\NormalTok{(dat[selected[}\DecValTok{1}\NormalTok{],],selected[}\DecValTok{1}\NormalTok{])}
\KeywordTok{plotImage}\NormalTok{(dat[selected[}\DecValTok{2}\NormalTok{],],selected[}\DecValTok{2}\NormalTok{])}
\KeywordTok{plotImage}\NormalTok{(dat[selected[}\DecValTok{3}\NormalTok{],],selected[}\DecValTok{3}\NormalTok{])\}}
\end{Highlighting}
\end{Shaded}

\includegraphics{eigenface_notebook_files/figure-latex/plot images-1.pdf}

\subsection{Metadata Import}\label{metadata-import}

The metadata comes from MIT already split into two sets, training and
testing.

\begin{Shaded}
\begin{Highlighting}[]
\NormalTok{meta_TR<-}\KeywordTok{importMetaMatrix}\NormalTok{(}\StringTok{"faces/faceDR"}\NormalTok{)}
\NormalTok{meta_T<-}\KeywordTok{importMetaMatrix}\NormalTok{(}\StringTok{"faces/faceDS"}\NormalTok{)}
\NormalTok{meta_TR<-meta_TR[meta_TR}\OperatorTok{$}\NormalTok{n }\OperatorTok\StringTok{ }\KeywordTok{rownames}\NormalTok{(dat),] }\CommentTok{# Remove any examples from metadata that are not in the data.}
\NormalTok{meta_T<-meta_T[meta_T}\OperatorTok{$}\NormalTok{n }\OperatorTok\StringTok{ }\KeywordTok{rownames}\NormalTok{(dat),]}
\NormalTok{dat_TR<-dat[meta_TR}\OperatorTok{$}\NormalTok{n,]}
\NormalTok{dat_T<-dat[meta_T}\OperatorTok{$}\NormalTok{n,]}
\CommentTok{#rm(dat)}
\end{Highlighting}
\end{Shaded}

\begin{center}\includegraphics{eigenface_notebook_files/figure-latex/metadata barplot-1} \end{center}

\subsection{Average faces}\label{average-faces}

\includegraphics{eigenface_notebook_files/figure-latex/cmeans-1.pdf}
\#\# ML with Caret

\begin{Shaded}
\begin{Highlighting}[]
\NormalTok{pca.res<-}\KeywordTok{prcomp}\NormalTok{(dat_TR,}\DataTypeTok{center=}\OtherTok{TRUE}\NormalTok{,}\DataTypeTok{rank. =}\NormalTok{ params}\OperatorTok{$}\NormalTok{n_features)}
\end{Highlighting}
\end{Shaded}

\includegraphics{eigenface_notebook_files/figure-latex/pca diagnostics-1.pdf}

\includegraphics{eigenface_notebook_files/figure-latex/plot eigen faces-1.pdf}
\includegraphics{eigenface_notebook_files/figure-latex/plot eigen faces-2.pdf}


\end{document}
